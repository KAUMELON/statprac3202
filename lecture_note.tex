% Options for packages loaded elsewhere
\PassOptionsToPackage{unicode}{hyperref}
\PassOptionsToPackage{hyphens}{url}
%
\documentclass[
]{book}
\usepackage{amsmath,amssymb}
\usepackage{lmodern}
\usepackage{iftex}
\ifPDFTeX
  \usepackage[T1]{fontenc}
  \usepackage[utf8]{inputenc}
  \usepackage{textcomp} % provide euro and other symbols
\else % if luatex or xetex
  \usepackage{unicode-math}
  \defaultfontfeatures{Scale=MatchLowercase}
  \defaultfontfeatures[\rmfamily]{Ligatures=TeX,Scale=1}
\fi
% Use upquote if available, for straight quotes in verbatim environments
\IfFileExists{upquote.sty}{\usepackage{upquote}}{}
\IfFileExists{microtype.sty}{% use microtype if available
  \usepackage[]{microtype}
  \UseMicrotypeSet[protrusion]{basicmath} % disable protrusion for tt fonts
}{}
\makeatletter
\@ifundefined{KOMAClassName}{% if non-KOMA class
  \IfFileExists{parskip.sty}{%
    \usepackage{parskip}
  }{% else
    \setlength{\parindent}{0pt}
    \setlength{\parskip}{6pt plus 2pt minus 1pt}}
}{% if KOMA class
  \KOMAoptions{parskip=half}}
\makeatother
\usepackage{xcolor}
\IfFileExists{xurl.sty}{\usepackage{xurl}}{} % add URL line breaks if available
\IfFileExists{bookmark.sty}{\usepackage{bookmark}}{\usepackage{hyperref}}
\hypersetup{
  pdftitle={PRACTICAL MANUAL OF STAT 3202},
  pdfauthor={Dr.~Pratheesh P. Gopinath \& Neethu S Kumar},
  hidelinks,
  pdfcreator={LaTeX via pandoc}}
\urlstyle{same} % disable monospaced font for URLs
\usepackage{longtable,booktabs,array}
\usepackage{calc} % for calculating minipage widths
% Correct order of tables after \paragraph or \subparagraph
\usepackage{etoolbox}
\makeatletter
\patchcmd\longtable{\par}{\if@noskipsec\mbox{}\fi\par}{}{}
\makeatother
% Allow footnotes in longtable head/foot
\IfFileExists{footnotehyper.sty}{\usepackage{footnotehyper}}{\usepackage{footnote}}
\makesavenoteenv{longtable}
\usepackage{graphicx}
\makeatletter
\def\maxwidth{\ifdim\Gin@nat@width>\linewidth\linewidth\else\Gin@nat@width\fi}
\def\maxheight{\ifdim\Gin@nat@height>\textheight\textheight\else\Gin@nat@height\fi}
\makeatother
% Scale images if necessary, so that they will not overflow the page
% margins by default, and it is still possible to overwrite the defaults
% using explicit options in \includegraphics[width, height, ...]{}
\setkeys{Gin}{width=\maxwidth,height=\maxheight,keepaspectratio}
% Set default figure placement to htbp
\makeatletter
\def\fps@figure{htbp}
\makeatother
\setlength{\emergencystretch}{3em} % prevent overfull lines
\providecommand{\tightlist}{%
  \setlength{\itemsep}{0pt}\setlength{\parskip}{0pt}}
\setcounter{secnumdepth}{5}
\usepackage{booktabs}
\usepackage{longtable}
\usepackage{array}
\usepackage{multirow}
\usepackage{wrapfig}
\usepackage{float}
\usepackage{colortbl}
\usepackage{pdflscape}
\usepackage{tabu}
\usepackage{threeparttable}
\usepackage{threeparttablex}
\usepackage[normalem]{ulem}
\usepackage{makecell}
\usepackage{xcolor}
\ifLuaTeX
  \usepackage{selnolig}  % disable illegal ligatures
\fi

\title{PRACTICAL MANUAL OF STAT 3202}
\author{Dr.~Pratheesh P. Gopinath \& Neethu S Kumar}
\date{2022-01-30}

\begin{document}
\maketitle

{
\setcounter{tocdepth}{1}
\tableofcontents
}
\hypertarget{welcome}{%
\chapter*{Welcome}\label{welcome}}
\addcontentsline{toc}{chapter}{Welcome}

Welcome to the book \textbf{PRACTICAL MANUAL OF STATISTICAL METHODS AND APPLICATIONS (STAT 3202)}.

\hypertarget{preface}{%
\chapter*{Preface}\label{preface}}
\addcontentsline{toc}{chapter}{Preface}

\textbf{Note}: This book is published in MeLoN (Module for e-Learning \& Online Notes) . The online version of this book is free to read here. You can download the PDF and take print out while submitting the records.

If you have any feedback, please feel free to contact \href{https://coavellayani.kau.in/people/dr-pratheesh-p-gopinath}{Dr.Pratheesh P. Gopinath}. E-mail: \emph{\href{mailto:pratheesh.pg@kau.in}{\nolinkurl{pratheesh.pg@kau.in}}}. Thank you!

This is a Practical Manual of STAT 3202 covering the syllabus of statistics course in B.Sc.(Hons.) Agriculture under Kerala Agricultural University

\hypertarget{exercise-1}{%
\chapter{Exercise 1}\label{exercise-1}}

\hypertarget{construction-of-frequency-distribution}{%
\section{Construction of Frequency Distribution}\label{construction-of-frequency-distribution}}

\underline{Steps in construction of frequency distribution}:

\begin{itemize}
\tightlist
\item
  Determine the number of classes\\
\item
  Determine the class width\\
\item
  Set up the individual class limits\\
\item
  Tally the items into the classes\\
\item
  Count the number of items in each class
\end{itemize}

Table 1.1 below shows the data related to the grain yield in (g /plot) of a sorghum variety from experimental plots of equal area from a continuous frequency distribution. Prepare a frequency distribution and cumulative frequency distribution

\begin{table}

\caption{\label{tab:t11}Frequency distribution }
\centering
\begin{tabular}[t]{r|r|r|r|r|r|r|r|r|r}
\hline
194 & 129 & 166 & 164 & 154 & 139 & 128 & 120 & 80 & 168\\
\hline
150 & 186 & 156 & 179 & 153 & 157 & 155 & 115 & 676 & 171\\
\hline
118 & 143 & 191 & 148 & 152 & 187 & 129 & 119 & 139 & 177\\
\hline
191 & 214 & 167 & 165 & 186 & 111 & 155 & 164 & 125 & 99\\
\hline
86 & 170 & 111 & 169 & 141 & 164 & 89 & 180 & 225 & 139\\
\hline
127 & 136 & 144 & 165 & 154 & 74 & 156 & 142 & 162 & 160\\
\hline
171 & 134 & 177 & 178 & 168 & 165 & 188 & 131 & 154 & 107\\
\hline
189 & 156 & 176 & 150 & 142 & 144 & 153 & 190 & 183 & 180\\
\hline
161 & 170 & 195 & 136 & 91 & 187 & 152 & 145 & 98 & 166\\
\hline
\end{tabular}
\end{table}

\underline{\textbf{SOLUTION}}:

Number of Classes, \emph{k} =1 + 3.322 log\textsubscript{10}\emph{N} =

Class width, C = \textbar{} max - min\textbar/ k =

Lower limit as L = min - \(\frac{c^{'} - k^{'} - (max - min)}{2}\) =

\hypertarget{exercise-2}{%
\chapter{Exercise 2}\label{exercise-2}}

\hypertarget{measures-of-central-tendency}{%
\section{Measures of Central Tendency}\label{measures-of-central-tendency}}

\begin{enumerate}
\def\labelenumi{\arabic{enumi}.}
\item
  If the weights of 5 ear heads of sorghum are 100, 102, 118, 124 \& 126, find the mean weight?
\item
  Calculate the mean value for the frequency distribution of weights of sorghum ear heads (Table 2.1)?\\
  \textbackslash begin\{table\}
\end{enumerate}

\caption{\label{tab:t1}mean calculation}
\centering
\begin{tabular}[t]{l|l|l|l|l|l|l|l|l}
\hline
Wt of ear head & 40-60 & 60-80 & 80-100 & 100-120 & 120-140 & 140-160 & 160-180 & 180-200\\
\hline
number & 6 & 28 & 35 & 55 & 30 & 15 & 12 & 9\\
\hline
\end{tabular}

\textbackslash end\{table\}
3. If the weight of sorghum ear heads are 45, 60, 48, 100 \& 65, find the median.\\
4. Find out the median and mode for the data in Table 2.1
5. Calculate the mode value for the following frequency distribution (Table 2.2)

\begin{table}

\caption{\label{tab:t2}mode calculation}
\centering
\begin{tabular}[t]{l|l|l|l|l|l|l|l|l}
\hline
C.I & 0-10 & 10-20 & 20-30 & 30-40 & 40-50 & 50-60 & 60-70 & 70-80\\
\hline
Freq. & 4 & 2 & 18 & 22 & 21 & 19 & 10 & 3\\
\hline
\end{tabular}
\end{table}

\begin{enumerate}
\def\labelenumi{\arabic{enumi}.}
\setcounter{enumi}{5}
\tightlist
\item
  Find out the mean, median and mode for weekly wages of 100 workers in a farm\\
  \textbackslash begin\{table\}
\end{enumerate}

\caption{\label{tab:t3}weekly wages}
\centering
\begin{tabular}[t]{l|r}
\hline
Weekly.wages & Number.of.workers\\
\hline
20-24 & 4\\
\hline
25-29 & 5\\
\hline
30-34 & 12\\
\hline
34-39 & 23\\
\hline
40-44 & 31\\
\hline
45-49 & 10\\
\hline
50-54 & 8\\
\hline
55-59 & 5\\
\hline
60-64 & 2\\
\hline
\end{tabular}

\textbackslash end\{table\}
7. The following data gives number of flowers observed from 20 plants.Find the arithmetic mean, geometric mean, harmonic mean.\\
42, 88, 37, 75, 98, 93, 73, 62, 96, 80, 52, 76, 66, 54, 73, 69, 83, 62, 53, 79\\
8. Table 2.4 below gives the distribution of the heights of 60 students in a Senior High school. Find Q1 \& Q3\\

\begin{table}

\caption{\label{tab:t4}height of students}
\centering
\begin{tabular}[t]{l|l|l|l|l|l|l}
\hline
Height & 145-150 & 150-155 & 155-160 & 160-165 & 165-170 & 170-175\\
\hline
No: of students & 3 & 9 & 16 & 18 & 10 & 4\\
\hline
\end{tabular}
\end{table}

\begin{enumerate}
\def\labelenumi{\arabic{enumi}.}
\setcounter{enumi}{8}
\tightlist
\item
  Find 𝑃25, 𝑃50\& 𝑃75 and D5 and Q2 for the Table 2.5 given below
  \textbackslash begin\{table\}
\end{enumerate}

\caption{\label{tab:t5}Frequency data}
\centering
\begin{tabular}[t]{l|r|r}
\hline
class & frequency & cf\\
\hline
0-10 & 11 & 11\\
\hline
Oct-20 & 18 & 29\\
\hline
20-30 & 25 & 54\\
\hline
30-40 & 28 & 82\\
\hline
40-50 & 30 & 112\\
\hline
50-60 & 33 & 145\\
\hline
60-70 & 22 & 167\\
\hline
70-80 & 15 & 182\\
\hline
80-90 & 12 & 194\\
\hline
90-100 & 10 & 204\\
\hline
\end{tabular}

\textbackslash end\{table\}

\hypertarget{references}{%
\chapter{References}\label{references}}

\end{document}
